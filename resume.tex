\documentclass[a4paper,10pt]{article}


    % Change the bottom margin
    \usepackage[bottom = 2.50cm]{geometry}

    %---------
    %- ATTRIBUTION:
    %-
    %- This is a modified version of the resume template found here:
    %-     http://www.cv-templates.info/2009/03/professional-cv-latex/
    %-

    \usepackage{marvosym}
    \usepackage{url,parskip}
    \usepackage{xunicode,xltxtra}

    \RequirePackage{color,graphicx}
    \usepackage[usenames,dvipsnames]{xcolor}
    \usepackage[big]{layaureo}          % better formatting of the A4 page
    %- Note: An alternative to Layaureo can be ** \usepackage{fullpage} **

    \usepackage{titlesec}               % custom \section

    % Note: for tabular alignment
    \usepackage{array}
    \newcolumntype{R}[1]{>{\raggedright\let\newline\\\arraybackslash\hspace{0pt}}p{ #1}}

    % -- Hyper-Linking --
    \usepackage{hyperref}
    \definecolor{linkcolour}{rgb}{0,0.2,0.6}
    \hypersetup{colorlinks,breaklinks,urlcolor=linkcolour, linkcolor=linkcolour}

    % -- Font Setup --
    \usepackage{fontspec}
    \defaultfontfeatures{Mapping=tex-text}
    \setmainfont[SmallCapsFont = Fontin SmallCaps]{Fontin}

    % CV Sections inspired by:
    %   http://stefano.italians.nl/archives/26
    \titleformat{\section}{\Large\scshape\raggedright}{}{0em}{}[\titlerule]
    \titlespacing{\section}{0pt}{5pt}{5pt}


\begin{document}


    \pagestyle{empty} % non-numbered pages

    % -- SECTION: TITLE --
    
        \par{\centering
            {\Huge Brandon Sandrowicz }
            \bigskip\par}
    

    % -- SECTION: PERSONAL DATA --
    
        \section{Personal Data}
        \begin{tabular}{ r l }
        \textsc{Citizenship}                    & United States\\
        \textsc{Immigration Status:}            & Permanent Resident (Canada)\\
        \textsc{Address:}                       & 25 Lambton Ave, Toronto, ON M6N 2S2\\
        \textsc{Phone:}                         & +1 647 960 3722\\
        \textsc{Email:}                         & \href{mailto:brandon@sandrowicz.org}{brandon@sandrowicz.org}\\
        \textsc{Github:}                        & \href{http://github.com/bsandrow}{github.com/bsandrow}\\
        \end{tabular}
    

    % -- SECTION: OBJECTIVE --
    
        \section{Objective}
        Find a job where I can do at least two of the following things: (i) work with
        interesting people, (ii) work on solving interesting problems, (iii) work on
        building something great. I am geo-bound to the Toronto area, and am
        not considering relocating at this time.
    

    % -- SECTION: WORK EXPERIENCE --
    
        \section{Work Experience}
        \begin{tabular}{ R{1.5cm} | p{10.5cm} }
            %---------
            %- Rentrak
            %-
            \hfill \textsc{Aug 2008}\newline
            \begin{center} to \end{center}
            \hfill \textsc{Nov 2012} &
                            Software Developer at Rentrak Corp.\newline
                            Portland, Oregon, USA and Toronto, Ontario, Canada\newline
                            \emph{Media data analysis and auditing} \newline
                            \fontsize{9pt}{10pt}\selectfont
                            \begin{itemize}
                                \item Developed cross-platform application in Groovy to interact with POS
                                    systems and transmit transaction information back to central server.
                                \item Developed internal CRM to facilitate multiple business units' interactions
                                    with customers, including an email gateway, threaded comments, and porting of
                                    legacy CRM data.
                                \item Maintained legacy internal applications (including GUIs in Java 1.3, and
                                    'green screens' in C), including adding new features as required.
                                \item Ported competing system (obtained via acquisition) within existing systems,
                                    including data import, reporting system changes and data warehousing.
                                \item Optimized database access as volume passed thresholds that reduced
                                    performance to unacceptable levels.
                                \item Implemented marketing materials admin interface in HTML/JS, including
                                    drag-and-drop sorting.
                                \item Created internal release tool on top of git with Python/Django.
                            \end{itemize}
                            \bigskip
                            \begin{description}
                                \item[Operating Systems:] Linux (CentOS), MS Windows
                                \item[Databases:] Unify, Oracle, PostgreSQL
                                \item[Languages:] Java, C, Groovy, Perl, Python, Pro*C
                                \item[Software/Tools:] Vim, Emacs, Apache, Nginx, Django, SQL*Plus
                            \end{description}
                                    \\
        \end{tabular}
    

    % -- SECTION: EDUCATION --
    
        \section{Education}
        \begin{tabular}{rl}
        \textsc{Aug} 2005 & Bachelor of Computer Science in \textsc{Engineering}, \textbf{University of Michigan}\\
        \end{tabular}
    

    % -- SECTION: TECHNICAL SKILLS --
    
        \section{Technical Skills}
        \begin{tabular}{ r p{11cm} }
        Dabbled in: & \textsc{Haskell} \symbol{"2022}
                      \textsc{PHP} \symbol{"2022}
                      \textsc{MatLab} \symbol{"2022}
                      \textsc{Lisp} \symbol{"2022}
                      \textsc{C++} \\
        Worked with: & \textsc{C} \symbol{"2022} \textsc{Java} \symbol{"2022}
                                \textsc{Oracle} \symbol{"2022} \textsc{PostgreSQL} \symbol{"2022}
                                \textsc{sh/bash/zsh} \symbol{"2022} \textsc{Javascript} \symbol{"2022}
                                \textsc{jQuery} \symbol{"2022} \textsc{Chrome Extensions} \symbol{"2022}
                                \textsc{CentOS}
                        \\
        Use everyday: & \textsc{Python} \symbol{"2022}
                        \textsc{git} \symbol{"2022}
                        \textsc{HTML} \symbol{"2022}
                        \textsc{Perl} \symbol{"2022}
                        \textsc{Linux} \symbol{"2022}
                        \textsc{Ubuntu} \\
        \end{tabular}
    

    % -- SECTION: HOBBIES --
    
    



\end{document}

% vim:ft=tex